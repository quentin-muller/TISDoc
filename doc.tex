\section{applyLUT}
\begin{lstlisting}[style=docstringstyle]

    applyLUT(f,lut)
    
    Applique la LookUpTable sur une image f
    
    Parameters
    ----------
    f    : Image de base en uint8
    lut  : Look Up Table

    Returns
    -------
    g   : Image transformée par la LUT 
    
\end{lstlisting}
\section{computeCumulativeHisto}
\begin{lstlisting}[style=docstringstyle]

    computeCumulativeHisto(h)
    
    Calcul l'histogramme cumulatif non normalisé
    et normalisé à partir de l'histogramme
    
    Parameters
    ----------
    h  : Histogramme de l'image

    Returns
    -------
    cumulativeHisto           : Histogramme cumulatif 
    normalizedCumulativeHisto : Histogramme cumulatif normalisé
    
\end{lstlisting}
\section{computeHisto}
\begin{lstlisting}[style=docstringstyle]

    computeHisto(f)
    
    Calcul l'histogramme d'un tableau 2D en uint8
    
    Parameters
    ----------
    f  : tableau 2D à calculer 

    Returns
    -------
    h : Histogramme de l'image f
    
\end{lstlisting}
\section{dft}
\begin{lstlisting}[style=docstringstyle]

    Discrete Fourier transform matrix.

    Create the matrix that computes the discrete Fourier transform of a
    sequence [1]_.  The n-th primitive root of unity used to generate the
    matrix is exp(-2*pi*i/n), where i = sqrt(-1).

    Parameters
    ----------
    n : int
        Size the matrix to create.
    scale : str, optional
        Must be None, 'sqrtn', or 'n'.
        If `scale` is 'sqrtn', the matrix is divided by `sqrt(n)`.
        If `scale` is 'n', the matrix is divided by `n`.
        If `scale` is None (the default), the matrix is not normalized, and the
        return value is simply the Vandermonde matrix of the roots of unity.

    Returns
    -------
    m : (n, n) ndarray
        The DFT matrix.

    Notes
    -----
    When `scale` is None, multiplying a vector by the matrix returned by
    `dft` is mathematically equivalent to (but much less efficient than)
    the calculation performed by `scipy.fftpack.fft`.

    .. versionadded:: 0.14.0

    References
    ----------
    .. [1] "DFT matrix", https://en.wikipedia.org/wiki/DFT_matrix

    Examples
    --------
    >>> from scipy.linalg import dft
    >>> np.set_printoptions(precision=5, suppress=True)
    >>> x = np.array([1, 2, 3, 0, 3, 2, 1, 0])
    >>> m = dft(8)
    >>> m.dot(x)   # Compute the DFT of x
    array([ 12.+0.j,  -2.-2.j,   0.-4.j,  -2.+2.j,   4.+0.j,  -2.-2.j,
            -0.+4.j,  -2.+2.j])

    Verify that ``m.dot(x)`` is the same as ``fft(x)``.

    >>> from scipy.fftpack import fft
    >>> fft(x)     # Same result as m.dot(x)
    array([ 12.+0.j,  -2.-2.j,   0.-4.j,  -2.+2.j,   4.+0.j,  -2.-2.j,
             0.+4.j,  -2.+2.j])
    
\end{lstlisting}
\section{halfToning}
\begin{lstlisting}[style=docstringstyle]

    halfToning(f)
    
    Transforme une image en nuance de gris en une image noir et blanc 
        ayant une moyenne identique à l'image de base
    
    Parameters
    ----------
    f     : Image de base en uint8

    Returns
    -------
    ht   : Image ajustée en tout ou rien
    
\end{lstlisting}
\section{imgLevelAdjust}
\begin{lstlisting}[style=docstringstyle]

    imgLevelAdjust(f, _mini=1, _maxi=99, debug=False)
    
    Ajuster l'histogramme de l'image pour utiliser toute la plage de l'uint8
    
    Parameters
    ----------
    f     : Image de base en uint8
    _mini : Limite minimum en %
    _maxi : Limite maximum en %
    debug : Affichage des print pour debug

    Returns
    -------
    h   : Image ajustée pour valeur min et max
    
\end{lstlisting}
\section{showImage}
\begin{lstlisting}[style=docstringstyle]

    showImage(Image, width=10, color='Greys_r', showGrid=True, gridTicks=1, 
        labelTicks=0, title= "Image", HLines=None, VLines=None)
    
    Afficher une image 2D sur un plot avec les paramètres suivants
    
    Parameters
    ----------
    Image      : tableau 2D à afficher  
    width      : Largeur du plot
    color      : Couleur du plot   
    showGrid   : Afficher ou non la grille  
    gridTicks  : Définir manuellement le pas de la grille
    labelTicks : Définir manuellement le pas du label
    title      : Titre de l'image  
    Hlines     : Ajoute une line horizontale à la hauteur H
    Vlines     : Ajoute une line verticale à la hauteur V
      
    Returns
    -------
    out :  None
    
\end{lstlisting}
